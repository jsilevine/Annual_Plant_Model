

% Annual Plant Hydraulic Model - Documentation
% This article will contain all documentation related to the model, including
% derivations, parameter choice, etc

% preamble
\documentclass{article}
\usepackage{graphicx}
\usepackage{amsmath}
\usepackage{breqn}
\usepackage{booktabs}
\usepackage{array}
\graphicspath{ {./figures/} }

%title construciton
\title{Hydraulic Competition in Annual Plants - Model Documentation}
\author{Jacob Levine, Steve Pacala and Jonathan Levine}
\date{}

\begin{document}

\maketitle

  \tableofcontents

  % Introduction section - motivate problem, introduce system
  \section{Introduction and Disclaimer}

  This article contains documentation pertaining to a model for hydraulic
  competition in annual plants. The primary aim of the document is to serve as an
  evolving record of the work done on the model, and to more-or-less
  informally communicate that work to co-authors. Therefore, the tone and
  structure of what follows will be slightly different than what you would
  expect in a document bound for the appendix of a manuscript. \p

  Given the emphasis on clear communication, this document likely
  includes more ``steps'' in the mathematical derivations and descriptions than
  would normally be desirable in a concise manuscript. Its also likely that my
  wording is frequently imprecise. This is more a reflection of my lack of
  experience communicating math than any kind of intentional
  decision, so please provide corrections where you see fit. \p

  I have tried to go through the model in a way that accumulates parameters gradually, but
  given the greek alphabet soup in this model (a compulsory element of plant
  physiological models), I have provided a Parameter Index in \ref{parameter dictionary}.

  % Introduce basic model elements and language
  \section{Physiological Model}

  \subsection{Model Description} \label{model description}

  The physiological portions of this model describe basic plant processes and
  thus relate relevant environmental quantities to the plant growth rates. There
  are three equations that make up the physiological portion of the model: one
  which deals entirely with carbon production via photosythesis, and two
  which relate photosynthesis to plant and soil hydrology. /p

  This first equation is a partially simplified version of Farquhar's photosynthesis model
  (citation), and relates a plant's carbon assimilation rate (minus root respiration $r_{l}$), $a$, to the atmospheric and internal concentrations of
  carbon ($C_{a}$ and $C_{i}$ respectively), as well as light availability
  ($V_{max}$). $\omega$ is an empirically determined constant.

  % Farquhar type photosynthesis model

  \begin{equation} \label{eq:Farquhar}
    a = \frac{VC_{i}}{\omega + C_{i}} - r_{l}
  \end{equation}

  The second pertinent relationship is an adaptation of Fick's law, and
  describes the dependence of a plant's carbon assimilation rate on the
  plant's stomatal conductance ($g$) and the gradient of external to
  internal carbon concentration.

  \begin{equation} \label{eq:Fick's Law}
    a = g[C_{a} - C_{i}]
  \end{equation}

  where

  $$g = \frac{m a \beta(\psi_{l})}{(C_{a} - \Gamma)(1 + \frac{D_{s}}{D_{0}})}$$

  Here, $m$ and $D_{0}$ are empirical constants. $D_{s}$ is the vapor pressure
  deficit, and $\Gamma$ is the $CO_{2}$ compensation point. $\psi_{l}$ is the leaf
  water potential, and $\beta(\psi_{l})$ is a shutoff operator bounded by 0 and
  1 which goes to 0 as $\psi_{l}$ goes to the wilting point. When the plant is
  not water limited, $\beta(\psi_{l}) = 1$. For convenience, in the future we
  will label the constant portion of $g$, $\phi$. Therefore, $g$ becomes:

  $$g = \phi a \beta(\psi_{l})$$

  Finally, we use two equations for evapotranspiration: one which is a function
  of leaf area and another which is a function of root area. In the leaf
  area equation, evapotranspiration per unit leaf area is dependent on the plant
  stomatal conductance, $g$, and the vapor pressure deficit, $D_{s}$. In the
  root area equation, evapotranpiration is dependent on the xylem
  conductivity, $k$, and the water potential gradient from the soil to the leaf.
  Note that this equation is also a form of Fick's law.

  \begin{equation} \label{eq:Evapotranspiration}
    E = LgD_{s} = Rk(\psi_{s} - \psi_{l})
  \end{equation}

  \subsection{Calculating the carbon assimilation rate} \label{calculating a}

  To determine a plant's growth rate as a function of soil
  water content, we must first determine the relationship between soil water potential
  and carbon uptake, $a$. To do so, we first note that when a plant's leaf water
  potential is above the wilting point, the plant is not water limited, and
  therefore operates at the maximum possible carbon assimilation rate, $a_{max}$. The wilting point of the plant is the water potential at which
  the plant's stomates first begin to shut. We label this value $\psi_{0}$.

  We define $\psi_{s}^{*}$ as the soil water potential at which $\psi_{l}$ first reaches
  $\psi_{0}$. Therefore, for $\psi_{s} > \psi_{s}^{*}$:

  $$a = g(C_{a} - C_{i})$$

  where, recalling that at $a_{max}$, $ \beta(\psi_{l}) = 1 $:

  $$a = \phi a (C_{a} - C_{i})$$

  therefore,

  $$C_{i} = C_{a} - \frac{1}{\phi}$$

  Plugging this expression into equation \ref{eq:Farquhar}, we get


  \begin{equation} \label{eq:amax}
    a_{max} = \frac{V \left( C_{a} - \frac{1}{\phi} \right) }{ \omega + \left( C_{a} - \frac{1}{\phi} \right) } - r_{l}
  \end{equation}
  

  Now, to solve for $\psi_{s}^{*}$, we plug $\psi_{0}$ and $a_{max}$ into
  equation \ref{eq:Evapotranspiration}, yielding

  $$L \phi a_{max} D_{s} = R k (\psi_{s}^{*} - \psi_{0})$$

  Therefore,

  \begin{equation} \label{eq:psistar}
    \psi_{s}^{*} = \frac{1}{\gamma} \frac{a_{max}D_{s}\phi}{k} + \psi_{0}
  \end{equation}

  where $\gamma$ is the root-shoot ratio of the plant.

  To calculate a plant's carbon assimilation rate during the period
  $\psi_{0} < \psi_{s} < \psi_{s}^{*}$, we solve the system of equations 1-3 for
  a, noting that during this period $\psi_{l} = \psi_{0}$.

  First, we can rearrange equation \ref{eq:Farquhar} to get

  $$C_{i} =  \frac{\omega(a + r_{l})}{V - r_{l} - a}$$

  Then, we can rearrange equation \ref{eq:Evapotranspiration} to get

  $$g = \frac{\gamma k}{D_{s}} (\psi_{s} - \psi_{0})$$

  and plug both into equation \ref{eq:Fick's Law} to get

  \begin{equation} \label{eq:a}
    (\psi_{s} - \psi_{0}) = \frac{a( V - r_{l} - a)}{\frac{\gamma k}{D_{s}} [C_{a}(V - r_{l} - a) + \omega(a + r_{l}) ]}
  \end{equation}

  By subtracting over the l.h.s and finding
  the roots of equation \ref{eq:a}, we can now solve $a$ for a given $\psi_{s}$.

  \subsection{A step-function approximation for a}

  The dynamics described in section \ref{calculating a} make the description of
  an analytical system for water competition impossible, even in the relatively
  simple case of mediterranean annuals. This is due to the nonlinear
  relationship between $\psi_{s}$ and $a$. This nonlinearity is further
  complicated by the relationship between volumetric water content, $W$, and
  soil water potential:

  \begin{equation} \label{Wfun}
    \frac{W - W_{min}}{W_{max} - W_{min}} = - \left( \frac{\psi_{min}}{\psi_{s}} \right)^{\lambda}
  \end{equation}

  However, if we plot $a$ as a function of $W$ across the seasonal range of water
  contents, we see that this relationship is nearly step-like, with all
  of the nonlinearity compressed within a small period at the end of the growing season.

  \begin{figure}[h]
    \caption{$a$ as a function of $W$, with $k = 2 kg m^{-1} MPa^{-1} season^{-1}$}
    \centering
    \includegraphics[width = 1\textwidth, height = 8cm]{a_kbig.png}
    \label{fig:a2}
  \end{figure}

  Figure \ref{fig:a2} shows $a$ as a function of $W$ with a xylem
  conductivity of $k = 2 kg m^{-1} MPa^{-1} season^{-1}$. Choices for the other parameter
  values were based off empirical estimates from Leuning et al. 1995 and can be
  found in section \ref{variable dictionary}. With $k = 2$, the relationship is
  nearly square, exhibiting an imperceptible degree of nonlinearity. This
  indicates that we can safely approximate the relationship between $a$ and
  $W$ as a step function, which in turn allows us to develop an analytical
  system by noting that with a step-function, plants operate at $a_{max}$ until the soil reaches
  $\psi_{0}$, at which point the plant stops growing for the season and converts
  its biomass to seed.

  \begin{figure}[h]
    \caption{$a$ as a function of $W$, with $k = 0.05 kg m^{-1} MPa^{-1} season^{-1}$}
    \centering
    \includegraphics[width = 1\textwidth, height = 8cm]{a_kmed.png}
    \label{fig:a0.05}
  \end{figure}

  To illustrate the robustness of this approximation, we can tweak $k$
  in an effort to make the function as ``curvy as possible''. Decreasing
  $k$ to $0.05 kg m^{-1} MPa^{-1} season^{-1}$, an already unrealistically
  small number, we begin to see an ever so slight blip of ``curviness'' in the
  relationship (Fig. \ref{fig:a0.05})

  \begin{figure}[h]
    \caption{$a$ as a function of $W$, with $k = 0.003 kg m^{-1} MPa^{-1} season^{-1}$}
    \centering
    \includegraphics[width = 1\textwidth, height = 8cm]{a_ksmall.png}
    \label{fig:a0.003}
  \end{figure}

  At the extreme, we can crank $k$ all the way down to
  $0.003 kg m^{-1} MPa^{-1} season^{-1}$, and see some amount of curve to the
  relationship (Fig. \ref{fig:a0.003}). However, even at this unrealistic value,
  the overestimation of $a$ that would occur is slight, especially when one notes that the rate of
  decrease in $W$ is largest in this region given it occurs when plants are near
  the season-maximal size, and are therefore extracting water much faster than
  they are at the beginning of the season.

  % detail allometric components of model including derivations
  \section{Allometry}

  \subsection{Model Description} \label{allometric model description}

  In order to calculate growth in the relevant dimensions, we will ascribe a
  power law allometry for annual plants.

  We are primarily interested in two dimensional quantities of the annual
  plants: leaf area and biomass. We are interested in leaf area because it is
  used to calculate evapotranspiration and photosynthetic rates, and because
  through the constant $\gamma$ it gives us the root area. We are
  interested in biomass because the plants' reproduction is proportional to it
  through the fecunditity constant $F$.

  Leaf area and biomass are related to each other by the following power law:

  \begin{equation} \label{power allometry}
    B = \delta L^{\nu}
  \end{equation}

  Where $\delta$ and $\nu$ are empirically determined constants.

  \subsection{Derivation of the Growth Rate Expression} \label{G derivation}

  To calculate a plant's growth rate, we begin by noting
  that to conserve carbon, the net carbon assimilation rate must
  equal the change in biomass, $B$, over time

  $$ L a_{max} - \gamma L r_{r} = \frac{dB}{dt}$$

  Where $r_{r}$ is the rate of root respiration per unit root area, recall that
  we have already accounted for leaf respiration. Next we want to solve for the
  growth rate in leaf area: $\frac{dL}{dt}$. To do so, we first note that

  $$ \frac{dB}{dt} = \frac{dB}{dL} \frac{dL}{dt} = \nu \delta L^{\nu - 1} \frac{dL}{dt} $$

  Solving for $\frac{dL}{dt}$ we get

  $$ \frac{dL}{dt} = \frac{L^{2 - \nu} ( a_{max} - \gamma r_{r})}{\nu \delta} $$

  For the system to be analytical, we require that the growth rate be
  constant (i.e. that $\frac{dL}{dt}$ is not a function of $L$). As it stands, this is not the
  case for $\frac{dL}{dt}$, so we define another, arbitrary plant dimension,
  $x$, which is related to $L$ in the following manner:

  $$ L = x^{\alpha} $$

  We then plug this equation for L into our expression for $\frac{dL}{dt}$ to
  obtain

  $$ \frac{dL}{dx}\frac{dx}{dt} = \frac{x^{\alpha(2 - \nu)}(a_{max} - \gamma r_{r})}{\nu \delta} $$

  Noting that $\frac{dL}{dx} = \alpha x^{\alpha - 1}$, we can solve for
  $\frac{dx}{dt}$, yielding

  $$ \frac{dx}{dt} = \frac{x^{\alpha \nu - 3\alpha + 1}(a_{max} - \gamma r_{r})}{\alpha \nu \delta} $$

  Now, to make $\frac{dx}{dt}$ constant, we can set
  $\alpha \nu - 3 \alpha + 1 = 0$, finding $alpha = \frac{1}{\nu - 1}$, thereby
  giving us the growth rate

  \begin{equation} \label{eq:Gx}
    G = \frac{dx}{dt} = \frac{\nu - 1}{\nu}\frac{a_{max} - \gamma r_{r}}{\delta}
  \end{equation}

  which implies
  \begin{equation} \label{eq:Gl}
    G_{L} = \frac{dL}{dt} = \left( \frac{\nu - 1}{nu} \frac{a_{max} - \gamma r_{r}}{\delta} \right)^{\frac{1}{\nu - 1}}
  \end{equation}

  Using this growth rate, we can calculate the total leaf area for species $i$ at
  time $t$, in year $T$ by the following relationship

  \begin{equation} \label{eq:L(t)}
    L(t) = N_{i, T}G_{i}^{\frac{1}{\nu - 1}}t^{\frac{1}{\nu - 1}}
  \end{equation}

  \subsection{Tradeoff between drought tolerance and growth rate} \label{G tradeoffs}




  % write out full system for no-depth case
  \section{The full system}

  \subsection{Quittin' Time}

  Consider a community of $Q$ plant species growing together and competing for
  water. We label the species from $1$ to $Q$ so that species $1$ has the
  highest $\psi_{0}$, and species $Q$ has the lowest $\psi_{0}$. At the
  beginning of the season, soil volumetric water content is at field capacity,
  and is given by $W_{0}$. From equations \ref{eq:Evapotranspiration} and
  \ref{eq:L(t)} we get

  $$ W_{1} =  W_{0} - a_{max}D_{s}\phi \int_{0}^{t_{1}} \! \sum_{j = 1}^{Q}N_{j, T}G_{j}^{\frac{1}{\nu-1}}t^{\frac{1}{\nu-1}}dt$$

 Where $t_{1}$ is the time at which species 1 stops groing, and $W_{1}$ is the
 volumetric water content at $t_{1}$. Taking the integral with respect to $t$ we find

  $$ W_{1} =  W_{0} - a_{max}D_{s}\phi \left[ \frac{\nu - 1}{\nu}\sum_{j = 1}^{Q}N_{j, T}G_{j}^{\frac{1}{\nu-1}}t_{1}^{\frac{\nu}{\nu-1}}  - 0\right] $$

  solving for $t_{1}$

  \begin{equation} \label{eq:t1}
    t_{1} = \left[ \frac{\nu}{\nu-1} \frac{1}{a_{max}D_{s}\phi} \frac{W_{0} - W_{1}}{\sum_{j=1}^{Q}N_{j, T}G_{j}^{\frac{1}{\nu - 1}}} \right]^{\frac{\nu - 1}{\nu}}
  \end{equation}

  Now, for $W_{i \neq 1}$ we write:

  $$ W_{i} =  W_{i-1} - a_{max}D_{s}\phi \int_{t_{i-1}}^{t_{i}} \! \sum_{j = i}^{Q}N_{j, T}G_{j}^{\frac{1}{\nu-1}}t^{\frac{1}{\nu-1}}dt$$

  which after taking the integral is

  $$ W_{i} =  W_{0} - a_{max}D_{s}\phi \left[ (\frac{\nu - 1}{\nu}) s\sum_{j = i}^{Q}N_{j, T}G_{j}^{\frac{1}{\nu-1}}t_{i}^{\frac{\nu}{\nu-1}}  -  \sum_{j = i}^{Q}N_{j, T}G_{j}^{\frac{1}{\nu-1}}t_{i-1}^{\frac{\nu}{\nu-1}}\right] $$

  and finally, solving for $t_{i}$ gives

  \begin{equation} \label{eq:ti}
    t_{i} = \left[ (\frac{\nu}{\nu-1}) (\frac{1}{a_{max}D_{s}\phi}) \sum_{k = 1}^{i}\frac{W_{k -1} - W_{k}}{\sum_{j=k}^{Q}N_{j, T}G_{j}^{\frac{1}{\nu - 1}}} \right]^{\frac{\nu - 1}{\nu}}
  \end{equation}

  \subsection{Population Dynamics}

  Plant reproduction is proportional to biomass by the fecundity constant $F$
  which can be interpreted as the number of germinating offspring produced per
  unit biomass. Thereby, using equations \ref{eq:Gx} and \ref{power allometry}
  we can write the expression for $N_{i, T+1}$ as:

  \begin{equation} \label{eq:NT+1}
    N_{i, T+1} = F N_{i, T} G_{i}^{\frac{\nu}{\nu - 1}}\left[ (\frac{\nu}{\nu-1})(\frac{1}{a_{max}D_{s}\phi}) \sum_{k = 1}^{i}\frac{W_{k -1} - W_{k}}{\sum_{j=k}^{Q}N_{j, T}G_{j}^{\frac{1}{\nu - 1}}} \right]
  \end{equation}

  \subsection{Equilibrium Abundances}

  First note that the sum in the denominator of species 1's population dynamics
  expression (equation \ref{eq:NT+1}) occurs in each of species 2-Q's population
  dynamics equations. This is also true for the sum in species 2's expression
  and species 3-Q's expressions and so on. Therefore by solving for this sum in
  the expression for species two, we can iteratively substitute in the solution
  and then solve for $N_{Q}^{*}$.

  At equilibrium $N_{i, T+1} = N_{i, T}$. Plugging this in to
  equation \ref{eq:NT+1} we obtain

  $$ 1 = F(\frac{\nu}{\nu - 1})(\frac{1}{a_{max}D_{s}\phi})G_{1}^{\frac{\nu}{\nu-1}}\left[ \frac{W_{0} - W_{1}}{\sum_{j = 1}^{Q}N_{j,T}G_{j}^{\frac{1}{\nu-1}}} \right] $$


  which can be solved for the sum in the denominator to get

  \begin{equation} \label{eq:equil_sum1}
    \sum_{j = 1}^{Q}N_{j,T}^{*}G_{j}^{\frac{1}{\nu-1}} = F(\frac{\nu}{\nu - 1})(\frac{1}{a_{max}D_{s}\phi})G_{1}^{\frac{\nu}{\nu - 1}}[W_{0} - W_{1}]
  \end{equation}

  Equation \ref{eq:eqil_sum1} can then be plugged into the expression for $N_{2, T+1}$ and
  solved for $\sum_{j = 2}^{Q}N_{j,T}^{*}G_{j}^{\frac{1}{\nu-1}}$, and so on to
  obtain the following generic expression for
  $\sum_{j = i}^{Q}N_{j, T}^{*}G_{j}^{\frac{1}{\nu - 1}}$:

  \begin{dmath} \label{eq:equil_sum}
    \sum_{j = i}^{Q}N_{j, T}^{*}G_{j}^{\frac{1}{\nu - 1}} = F(\frac{\nu}{\nu - 1})(\frac{1}{a_{max}D_{s}\phi})G_{i}^{\frac{\nu}{\nu - 1}}\left(\frac{G_{i-1}^{\frac{\nu}{\nu-1}}}{G_{i-1}^{\frac{\nu}{\nu-1}} - G_{i}^{\frac{\nu}{\nu-1}}} \right)(W_{i-1} - W_{i})
  \end{dmath}

  which in turn can be used to solve for the equilibrium abundance of
  species Q:

  \begin{equation} \label{eq:NstarQ}
   N_{Q, T}^{*} = F(\frac{\nu}{\nu - 1})(\frac{1}{a_{max}D_{s}\phi})(W_{Q-1} - W_{Q})G_{Q}(\frac{G_{Q-1}^{\frac{\nu}{\nu-1}}}{G_{Q-1}^{\frac{\nu}{\nu-1}} - G_{Q}^{\frac{\nu}{\nu-1}}})
 \end{equation}

  This expression can then be used to solve for the equilibrium population of
  species $i \neq 1, Q$:

  \begin{dmath} \label{eq:Nstari}
    N_{i \neq 1,Q; T}^{*} = F(\frac{\nu}{\nu - 1})(\frac{1}{a_{max}D_{s}\phi})G_{i}\left[ (W_{i-1} - W_{i})(\frac{G_{i-1}^{\frac{\nu}{\nu-1}}}{G_{i-1}^{\frac{\nu}{\nu-1}} - G_{i}^{\frac{\nu}{\nu-1}}}) - (W_{i} - W_{i+1})(\frac{G_{i+1}^{\frac{\nu}{\nu-1}}}{G_{i}^{\frac{\nu}{\nu-1}} - G_{i+1}^{\frac{\nu}{\nu-1}}}) \right]
  \end{dmath}

  and finally:

  \begin{equation} \label{eq:Nstar1}
    N_{1, T}^{*} = F(\frac{\nu}{\nu - 1})(\frac{1}{a_{max}D_{s}\phi})G_{1}\left[ (W_{0} - W_{1}) - (W_{1} - W_{2})(\frac{G_{2}^{\frac{\nu}{\nu-1}}}{G_{1}^{\frac{\nu}{\nu-1}} - G_{2}}) \right]
  \end{equation}


  \subsection{Invasion of a Rare Type} \label{invasion when rare}

  We consider a species, $q$, growing from rare abundance in a system of $Q-1$
  other species (labelled 1, 2, ..., q-1, q+1, ..., Q-1) at equilibrium
  abundance. By definition, the growth rate of species
  $q$ lies between the growth rates of species $q-1$ and $q+1$. Note that following
  equation \ref{eq:NT+1}, the population growth rate of any species, $r_{i}$, is given by

  $$ r_{i} =  c g_{i}\left[  \sum_{k = 1}^{i}\frac{W_{k -1} - W_{k}}{\sum_{j=k}^{Q}N_{j, T}G_{j}^{\frac{1}{\nu - 1}}} \right] $$

  where $c = F(\frac{\nu}{\nu-1})(\frac{1}{a_{max}D_{s}\phi})$, and $g = G^{\frac{\nu}{\nu-1}}$. Given that all species $i \neq q$ are at equilibrium abundance in this system,
  we can write the expression for $r_{q}$ as

  $$ r_{q} = c g_{q}\left[  \sum_{k = 1}^{q-1}\frac{W_{k -1} - W_{k}}{\sum_{j=k}^{Q}N_{j, T}^{*}G_{j}^{\frac{1}{\nu - 1}} + N_{q, T}G_{q}^{\frac{1}{\nu - 1}}} + \frac{W_{q-1} - W_{q}}{\sum_{j = q+1}^{Q}N_{k, T}^{*}G_{k}^{\frac{1}{\nu - 1}} + N_{q, T}G_{q}^{\frac{1}{\nu - 1}}} \right] $$

  Plugging in equations \ref{eq:equil_sum1} and \ref{eq:equil_sum} we get

  \begin{dmath*}
     r_{q} = c g_{q} \left[ \frac{W_{0} - W_{1}}{c g_{1} (W_{0} - W_{1}) + N_{q, T}G_{q}^{\frac{1}{\nu - 1}} } + \sum_{k = 1}^{q-1}\frac{W_{k -1} - W_{k}}{c g_{k}(\frac{g_{k-1}}{g_{k-1} - g_{k}})(W_{k-1} - W_{k}) + N_{q, T}G_{q}^{\frac{1}{\nu - 1}}} + \frac{W_{q-1} - W_{q}}{c g_{q+1}\frac{g_{q-1}}{g_{q-1} - g_{q+1}}(W_{q-1} - W_{q+1}) + N_{q, T}G_{q}^{\frac{1}{\nu - 1}}} \right]
  \end{dmath*}

  Now, to obtain the expression for $r_{q}$ at low abundance, we take
  $\lim_{N_{q, T}\to0}\! r_{q}$

  \begin{dmath*}
  \lim_{x\to0}r_{q} = c g_{q} \left[ \frac{W_{0} - W_{1}}{c g_{1} (W_{0} - W_{1}) + 0} + \sum_{k = 1}^{q-1}\frac{W_{k -1} - W_{k}}{c g_{k}(\frac{g_{k-1}}{g_{k-1} - g_{k}})(W_{k-1} - W_{k}) + 0} + \frac{W_{q-1} - W_{q}}{\sum_{j = q+1}^{Q}N_{k, T}^{*}G_{k}^{\frac{1}{\nu - 1}} + 0} \right] $$
  \end{dmath*}

  noting that
  $\frac{1}{g_{k-1}} + \frac{g_{k-1} - g_{k}}{g_{k-1}g_{k}} = \frac{1}{g_{k}}$,
  the above expression simplifies to

  $$ r_{q, invasion} = \frac{g_{q}}{g_{q-1}}\left[ 1 + \frac{g_{q-1} - g_{q+1}}{g_{q+1}} (\frac{W_{q-1} - W_{1}}{W_{q-1} - W_{q+1}}) \right] $$

  Therefore, invasion of species $q$ when rare is governed by

  \begin{equation} \label{eq:invasion}
    N_{q, T+1} = N_{q, T}\frac{g_{q}}{g_{q-1}}\left[ 1 + \frac{g_{q-1} - g_{q+1}}{g_{q+1}} (\frac{W_{q-1} - W_{1}}{W_{q-1} - W_{q+1}}) \right]
  \end{equation}

  which implies that invasion when rare requires:

  $$ r_{q, invasion} =\frac{g_{q}}{g_{q-1}}\left[ 1 + \frac{g_{q-1} - g_{q+1}}{g_{q+1}} (\frac{W_{q-1} - W_{1}}{W_{q-1} - W_{q+1}}) \right] > 1 $$

  \subsection{Evolutionary Stable Strategies}


  

  % include details about extension to depth cases
  \section{Model Extensions}

  %Index of parameter symbols, names and values if applicable
  \section{Parameter Index} \label{variable dictionary}

  \begin{table}
    \begin{tabular}{ll}
      \toprule
      \textbf{Parameter} & \textbf{Definition} \\
      \midrule
      $V$ & light availability above the crown \\
      $C_{a}$ & atmospheric carbon concentration \\
      $C_{i}$ & internal plant carbon concentration \\
      $\omega$ & empirical constant in Farquhar photosynthesis model, eq.
                 1 \\
      $a_{max}$ & maximum rate of carbon uptake \\
      $g$ & plant stomatal conductace \\
      $m$ & empirical constant from Fick's law, eq. 2 \\
      $\psi_{l}$ & leaf water potential \\
      $\Gamma$ & $CO_{2}$ compensation point \\
      $D_{s}$ & vapor pressure deficit \\
      $D_{0}$ & empirical constant from Fick's law, eq. 2 \\
      $\phi$ & shorthand expression of constants from Fick's law, eq. 2 \\
      $L$ & leaf area \\
      $R$ & root area \\
      $k$ & xylem conductivity \\
      $\psi_{s}$ & soil water potential \\
      $r_{l}$ & leaf respiration rate \\
      $r_{r}$ & root respiration rate \\
      $\psi_{0}$ & wilting leaf water potential \\
      $\gamma$ & root area to shoot area ratio \\
      $W$ & volumetric water content of the soil \\
      $B$ & plant biomass \\
      $\delta$ & allometric coefficient for relationship between $L$ and $B$ \\
      $\nu$ & allometric exponent for relationship between $L$ and $B$ \\
      $x$ & arbitrary plant dimension in which growth is measured \\
      $alpha$ & allometric exponenet that relates $x$ to $L$ \\
      $G$ & plant growth rate in $x$ \\
      $G_{L}$ & plant growth rate in $L$ \\
      $t$ & timestep (within season) \\
      $T$ & season \\
      $N_{i, T}$ & abundance of species $i$ during season $T$ \\
      $F$ & number of germinating offspring produced per unit biomass \\
      $N^{*}$ & equilibirum abundance \\
      $c$ & shorthand for $F(\frac{\nu}{\nu-1})(\frac{1}{a_{max}D_{s}\phi})$ \\
      $g$ & shorthand for $G^{\frac{\nu}{\nu-1}}$ \\
      \bottomrule
    \end{tabular}
  \end{table}


\end{document}
